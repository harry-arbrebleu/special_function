\documentclass{report}
\usepackage{luatexja} % LuaTeXで日本語を使うためのパッケージ
\usepackage{luatexja-fontspec} % LuaTeX用の日本語フォント設定

% --- 数学関連 ---
\usepackage{amsmath, amssymb, amsfonts, mathtools, bm, amsthm} % 基本的な数学パッケージ
\usepackage{type1cm, upgreek} % 数式フォントとギリシャ文字k
\usepackage{physics, mhchem} % 物理や化学の記号や式の表記を簡単にする

% --- 表関連 ---
\usepackage{multirow, longtable, tabularx, array, colortbl, dcolumn, diagbox} % 表のレイアウトを柔軟にする
\usepackage{tablefootnote, truthtable} % 表中に注釈を追加、真理値表
\usepackage{tabularray} % 高度な表組みレイアウト

% --- グラフィック関連 ---
\usepackage{tikz, graphicx} % 図の描画と画像の挿入
% \usepackage{background} % ウォーターマークの設定
\usepackage{caption, subcaption} % 図や表のキャプション設定
\usepackage{float, here} % 図や表の位置指定

% --- レイアウトとページ設定 ---
\usepackage{fancyhdr} % ページヘッダー、フッター、余白の設定
\usepackage[top = 20truemm, bottom = 20truemm, left = 20truemm, right = 20truemm]{geometry}
\usepackage{fancybox, ascmac} % ボックスのデザイン

% --- 色とスタイル ---
\usepackage{xcolor, color, colortbl, tcolorbox} % 色とカラーボックス
\usepackage{listings, jvlisting} % コードの色付けとフォーマット

% --- 参考文献関連 ---
\usepackage{biblatex, usebib} % 参考文献の管理と挿入
\usepackage{url, hyperref} % URLとリンクの設定

% --- その他の便利なパッケージ ---
\usepackage{footmisc} % 脚注のカスタマイズ
\usepackage{multicol} % 複数段組
\usepackage{comment} % コメントアウトの拡張
\usepackage{siunitx} % 単位の表記
\usepackage{docmute}
% \usepackage{appendix}
% --- tcolorboxとtikzの設定 ---
\tcbuselibrary{theorems, breakable} % 定理のボックスと改ページ設定
\usetikzlibrary{decorations.markings, arrows.meta, calc} % tikzの装飾や矢印の設定

% --- 定理スタイルと数式設定 ---
\theoremstyle{definition} % 定義スタイル
\numberwithin{equation}{section} % 式番号をサブセクション単位でリセット

% --- hyperrefの設定 ---
\hypersetup{
  setpagesize = false,
  bookmarks = true,
  bookmarksdepth = tocdepth,
  bookmarksnumbered = true,
  colorlinks = false,
  pdftitle = {}, % PDFタイトル
  pdfsubject = {}, % PDFサブジェクト
  pdfauthor = {}, % PDF作者
  pdfkeywords = {} % PDFキーワード
}

% --- siunitxの設定 ---
\sisetup{
  table-format = 1.5, % 小数点以下の桁数
  table-number-alignment = center, % 数値の中央揃え
}



% --- その他の設定 ---
\allowdisplaybreaks % 数式の途中改ページ許可
\newcolumntype{t}{!{\vrule width 0.1pt}} % 新しいカラムタイプ
\newcolumntype{b}{!{\vrule width 1.5pt}} % 太いカラム
\UseTblrLibrary{amsmath, booktabs, counter, diagbox, functional, hook, html, nameref, siunitx, varwidth, zref} % tabularrayのライブラリ
\setlength{\columnseprule}{0.4pt} % カラム区切り線の太さ
\captionsetup[figure]{font = bf} % 図のキャプションの太字設定
\captionsetup[table]{font = bf} % 表のキャプションの太字設定
\captionsetup[lstlisting]{font = bf} % コードのキャプションの太字設定
\captionsetup[subfigure]{font = bf, labelformat = simple} % サブ図のキャプション設定
\setcounter{secnumdepth}{4} % セクションの深さ設定
\newcolumntype{d}{D{.}{.}{5}} % 数値のカラム
\newcolumntype{M}[1]{>{\centering\arraybackslash}m{#1}} % センター揃えのカラム
\DeclareMathOperator{\diag}{diag}
\everymath{\displaystyle} % 数式のスタイル
\newcommand{\inner}[2]{\left\langle #1, #2 \right\rangle}
\renewcommand{\figurename}{図}
\renewcommand{\i}{\mathrm{i}} % 複素数単位i
\renewcommand{\laplacian}{\grad^2} % ラプラシアンの記号
\renewcommand{\thesubfigure}{(\alph{subfigure})} % サブ図の番号形式
\newcommand{\m}[3]{\multicolumn{#1}{#2}{#3}} % マルチカラムのショートカット
\renewcommand{\r}[1]{\mathrm{#1}} % mathrmのショートカット
\newcommand{\e}{\r{e}} % 自然対数の底e
\newcommand{\Ef}{E_{\r{F}}} % フェルミエネルギー
\renewcommand{\c}{\si{\degreeCelsius}} % 摂氏記号
\renewcommand{\d}{\r{d}} % d記号
\renewcommand{\t}[1]{\texttt{#1}} % タイプライタフォント
\newcommand{\kb}{k_{\r{B}}} % ボルツマン定数
\renewcommand{\epsilon}{\varepsilon}
\newcommand{\fullref}[1]{\textbf{\ref{#1} \nameref{#1}}}
\newcommand{\reff}[1]{\textbf{図\ref{#1}}} % 図参照のショートカット
\newcommand{\reft}[1]{\textbf{表\ref{#1}}} % 表参照のショートカット
\newcommand{\refe}[1]{\textbf{式\eqref{#1}}} % 式参照のショートカット
\newcommand{\refp}[1]{\textbf{コード\ref{#1}}} % コード参照のショートカット
\renewcommand{\lstlistingname}{コード} % コードリストの名前
\renewcommand{\theequation}{\thesection.\arabic{equation}} % 式番号の形式
\renewcommand{\footrulewidth}{0.4pt} % フッターの線
\newcommand{\mar}[1]{\textcircled{\scriptsize #1}} % 丸囲み文字
\newcommand{\combination}[2]{{}_{#1} \mathrm{C}_{#2}} % 組み合わせ
\newcommand{\thline}{\noalign{\hrule height 0.1pt}} % 細い横線
\newcommand{\bhline}{\noalign{\hrule height 1.5pt}} % 太い横線

% --- カスタム色定義 ---
\definecolor{burgundy}{rgb}{0.5, 0.0, 0.13} % バーガンディ色
\definecolor{charcoal}{rgb}{0.21, 0.27, 0.31} % チャコール色
\definecolor{forest}{rgb}{0.0, 0.35, 0} % 森の緑色

% --- カスタム定理環境の定義 ---
\newtcbtheorem[number within = chapter]{mydef}{定義}{
  fonttitle = \gtfamily\sffamily\bfseries\upshape,
  colframe = forest,
  colback = forest!2!white,
  rightrule = 1pt,
  leftrule = 1pt,
  bottomrule = 2pt,
  colbacktitle = forest,
  theorem style = standard,
  breakable,
  arc = 0pt,
}{definition-ref}
\newtcbtheorem[number within = chapter]{myprop}{命題}{
  fonttitle = \gtfamily\sffamily\bfseries\upshape,
  colframe = blue!50!black,
  colback = blue!50!black!2!white,
  rightrule = 1pt,
  leftrule = 1pt,
  bottomrule = 2pt,
  colbacktitle = blue!50!black,
  theorem style = standard,
  breakable,
  arc = 0pt
}{proposition-ref}
\newtcbtheorem[number within = chapter]{myrem}{注意}{
  fonttitle = \gtfamily\sffamily\bfseries\upshape,
  colframe = yellow!20!black,
  colback = yellow!50,
  rightrule = 1pt,
  leftrule = 1pt,
  bottomrule = 2pt,
  colbacktitle = yellow!20!black,
  theorem style = standard,
  breakable,
  arc = 0pt
}{remark-ref}
\newtcbtheorem[number within = chapter]{myex}{例題}{
  fonttitle = \gtfamily\sffamily\bfseries\upshape,
  colframe = black,
  colback = white,
  rightrule = 1pt,
  leftrule = 1pt,
  bottomrule = 2pt,
  colbacktitle = black,
  theorem style = standard,
  breakable,
  arc = 0pt
}{example-ref}
\newtcbtheorem[number within = chapter]{definition}{Definition}{mydef}{definition-ref}
\newcommand{\dfref}[1]{{\bfseries\sffamily 定義 \ref{definition-ref:#1}}}
\newtcbtheorem[number within = chapter]{prop}{命題}{myprop}{proposition-ref}
\newcommand{\prref}[1]{{\bfseries\sffamily 命題 \ref{proposition-ref:#1}}}
\newtcbtheorem[number within = chapter]{rem}{注意}{myrem}{remark-ref}
\newcommand{\rmref}[1]{{\bfseries\sffamily 注意 \ref{remark-ref:#1}}}
\newtcbtheorem[number within = chapter]{ex}{例題}{myex}{example-ref}
\newcommand{\exref}[1]{{\bfseries\sffamily 例題 \ref{example-ref:#1}}}
% --- 再定義コマンド ---
% \mathtoolsset{showonlyrefs=true} % 必要な式番号のみ表示
\pagestyle{fancy} % ヘッダー・フッターのスタイル設定
% \chead{応用量子物性講義ノート} % 中央ヘッダー
% \rhead{}
% \fancyhead[R]{\rightmark}
% \renewcommand{\subsectionmark}[1]{\markright{\thesubsection\ #1}}
% \cfoot{\thepage} % 中央フッターにページ番号
% \lhead{}
% \rfoot{Yuto Masuda and Haruki Aoki} % 右フッターに名前
% \setcounter{tocdepth}{4} % 目次の深さ
\makeatletter
\@addtoreset{equation}{section} % セクションごとに式番号をリセット
\makeatother

% --- メタ情報 ---
\title{物理と特殊函数}
\date{\today}
\author{Haruki AOKI}

\begin{document}
  \begin{abstract}
    本テクストは特殊函数と量子力学の繋がりを説明する.
    量子力学の究極的な目標は,系の状態ベクトルを把握することである.
    量子力学の基本的な要請により,系の状態ベクトル$\ket{\psi}$は,
    \begin{align}
      \hat{H}\ket{\psi} = E\ket{\psi}
    \end{align}
    なる固有方程式に帰着する.
    状態ベクトルは,考えている系の情報を全て含むものであり,$\ket{\psi}$と系の状態が1対1対応する.
    例えば,電子はスピンと呼ばれる物理量を持ち,スピンは上向きか下向きの2種類しか存在しないことがStern-Gerlachの実験により知られている.
    Stern-Gerklachの実験のセットアップでは,電子の状態ベクトルは上向きスピンと下向きスピンの2つの基底で
  \end{abstract}
    \chapter{特殊函数}
    本章では数ある特殊函数のうち,Strum-Liouville演算子と呼ばれるクラスの演算子の固有函数を扱う.
    この演算子のクラスは極めて性質が良いことが知られている.
    \section{Hermite演算子}
      \begin{mydef}{Hermite演算子}{}
        内積が定義されている函数空間$V$の任意の元$f$,$g$に対して,
        \begin{align}
          \inner{f}{\mathcal{L}g} = \inner{\mathcal{L}f}{g}
        \end{align}
        なる演算子$\mathcal{L}$をHermite演算子という.
      \end{mydef}
    \section{Strum-Liouvillle演算子}
      \begin{mydef}{Strum-Liouville演算子}{}
        $a < b$として,$x\in \qty[a, b]$で定義された函数空間$V$を考える.$\rho(x)$を非負の実数函数として,$f, g\in V$に対して,
        \begin{align}
          \inner{f}{g} \coloneqq \int_{a}^{b} f^*(x)g(x)\rho(x)\dd{x}
        \end{align}
        なる内積を入れる.
        函数空間$V$上の演算子として$\mathcal{L}$を,
        \begin{align}
          \mathcal{L} \coloneqq \frac{1}{\rho(x)}\qty[\dv{x}\qty{p(x)\dv{x}} + q(x)]\label{strum-liouville-operator-def}
        \end{align}
        とする.
        \refe{strum-liouville-operator-def}なる形をした演算子をStrum-Liouvillle演算子という.
      \end{mydef}
      \begin{myprop}{Strum-Liouville演算子のHermite性}{}
        境界条件を,$\forall f \in V$について,
        \begin{align}
          \begin{dcases}
            f(a) = f(b) \\ 
            p(a)f'(a) = p(b)f'(b)
          \end{dcases}
        \end{align}
        とすると,
        \begin{align}
          \inner{f}{\mathcal{L}g} = \inner{\mathcal{L}f}{g}\label{hermite-def}
        \end{align}
        が成立する.
        \tcblower
        \begin{proof}
          内積の定義より,
          \begin{align}
            \inner{f}{\mathcal{L}g} &= \int_{a}^{b} f^*(x)\frac{1}{\rho(x)}\qty[\dv{x}\qty{p(x)\dv{x}g(x)} + q(x)g(x)]\rho(x)\dd{x} \\ 
            &= \int_{a}^{b} f^*(x)\qty[\dv{x}\qty{p(x)\dv{x}g(x)} + q(x)g(x)]\dd{x} \\ 
            &= \qty[f^*(x)g'(x)]_{a}^{b} - \int_{a}^{b}f'^*(x)p(x)g'(x)\dd{x} + \int_{a}^{b}f^*(x)q(x)g(x)\dd{x} \\ 
            &= \qty[f^*(x)p(x)g'(x)]_{a}^{b} - \qty[f'^*(x)p(x)g(x)]_{a}^{b} + \int_{a}^{b}g(x)\frac{1}{\rho(x)}\qty[\dv{x}\qty(p(x)\dv{x}f^*(x)) + q(x)f^*(x)]\rho(x)\dd{x} \\ 
            &= \qty[f^*(x)p(x)g'(x)]_{a}^{b} - \qty[f'^*(x)p(x)g(x)]_{a}^{b} + \inner{g}{\mathcal{L}f}^* \\ 
            &= \qty[f^*(x)p(x)g'(x)]_{a}^{b} - \qty[g(x)p(x)f'^*(x)]_{a}^{b} + \inner{\mathcal{L}f}{g} 
          \end{align}
          となる.第1項と第2項について,第1項に$f(a) = f(b)$を,第2項に$p(x)$が実数値函数であり$p(a)f'(a) = p(b)f'(b)$であることを用いると
          \begin{align}
            \qty[f^*(x)p(x)g'(x)]_{a}^{b} - \qty[f'^*(x)p(x)g(x)]_{a}^{b} = \qty{p(b)g'(b) - p(a)f(a)}f^*(a) - \qty{g(b) - g(a)}p(a)f'(a)
          \end{align}
          を得る.今度は,第1項に$p(x)$が実数値函数であり$p(a)g'(a) = p(b)g'(b)$であることを,第2項に$g(a) = g(b)$を用いれば,
          \begin{align}
            \inner{f}{\mathcal{L}g} = \inner{\mathcal{L}f}{g}
          \end{align}
          を得る.
        \end{proof}
      \end{myprop}
    \section{Hermite多項式}
    \section{Legendre多項式}
      \begin{mydef}{$\mathcal{L}_m$の定義}{}
        演算子$\mathcal{L}_m$を,
        \begin{align}
          \mathcal{L}_m \coloneqq \dv{x}\qty{(1 - x^2)\dv{x}} - \frac{m^2}{1 - x^2}
        \end{align}
        と定義する.ただし,$m\in \qty{0, 1, \cdots}$である.
        境界条件は,
          \begin{align}
            p(\pm 1)f^*(\pm 1)g'(\pm 1) = 0
          \end{align}
          とする.
      \end{mydef}
      \begin{myrem}{}{}
        $\mathcal{L}_m$は\refe{strum-liouville-operator-def}において,
        \begin{align}
          a &= -b = 1 \\ 
          \rho(x) &= 1 \\ 
          p(x) &= 1 - x^2 \\ 
          q(x) &= -\frac{m^2}{1 - x^2}
        \end{align}
        としたものであるから,
        \begin{align}
          \mathcal{L}_m = \dv{x}\qty{(1 - x^2)\dv{x}} - \frac{m^2}{1 - x^2}
        \end{align}
        となる.
        内積は,
        \begin{align}
          \inner{f}{g} = \int_{-1}^{1} f^*(x)g(x) \dd{x} \label{inner-prod-legendre}
        \end{align}
        となる.
      \end{myrem}
      \begin{myprop}{$\mathcal{L}_m$のHermite性}{}
        演算子$\mathcal{L}_m$はHermite演算子である.
        \tcblower
        \begin{proof}
          たとえば,$g'(x) = p(x)$となるように$g(x)$を定めれば,$f(1) = f(-1) = 0$となる.
          また,$f(x) = 1$となるように$f(x)$を定めれば,$p(1)g'(1) = p(-1)g'(1) = 0$となるので,前節で示した境界条件を満足する.
        \end{proof}
      \end{myprop}
      さて,Legendre多項式$P_n(x)$は$n$を非負整数として,
      \begin{align}
        \mathcal{L}_0P_n(x) &= -n(n + 1) P_n(x) \\ 
        \Leftrightarrow \dv{x}\qty{(1 - x^2)\dv{x}P_n(x)} &= -n(n + 1)P_n(x)\label{legendre-polynominal} 
      \end{align}
      なる$P_n(x)$のうち,$x = 0$周りで級数展開したもので,
      \begin{align}
        P_n(x) = \sum_{j = 0}^{\infty}u_jx^j\label{legendre-polynominal-u-expantion}
      \end{align}
      と書いたとき,
      \begin{align}
        u_n &= \frac{(2n)!}{2^n(n!)^2}\label{legendre-polynominal-u-expantion-n} \\ 
        u_{n + 1} &= 0\label{legendre-polynominal-u-expantion-np1}
      \end{align}
      なるものである.\refe{legendre-polynominal-u-expantion}を\refe{legendre-polynominal}に代入すると,
      \begin{align}
        \dv{x}\qty{(1 - x^2)\dv{x}\sum_{j = 0}^{\infty}u_jx^j} &= -n(n + 1)\sum_{j = 0}^{\infty}u_jx^j \\ 
        \Leftrightarrow \sum_{j = 0}^{\infty}ju_j\dv{x}\qty(x^{j - 1} - x^{j + 1}) &= -n(n + 1)\sum_{j = 0}^{\infty}u_jx^j \\ 
        \Leftrightarrow \sum_{j = 0}^{\infty}j(j - 1)u_jx^{j - 2} - \sum_{j = 0}^{\infty}j(j + 1)u_jx^j &= -n(n + 1)\sum_{j = 0}^{\infty}u_jx^j \\ 
        \Leftrightarrow \sum_{j = 0}^{\infty}j(j - 1)u_jx^{j - 2} &= \sum_{j = 0}^{\infty}u_j\qty[j(j + 1) - n(n + 1)]x^j \\ 
        \Leftrightarrow \sum_{j = 2}^{\infty}j(j - 1)u_jx^{j - 2} &= \sum_{j = 0}^{\infty}u_j\qty[j(j + 1) - n(n + 1)]x^j \\ 
        \Leftrightarrow \sum_{j = 0}^{\infty}(j + 1)(j + 2)u_{j + 2}x^{j} &= \sum_{j = 0}^{\infty}u_j\qty[j(j + 1) - n(n + 1)]x^j 
      \end{align}
      となるから,
      \begin{align}
        (j + 1)(j + 2)u_{j + 2} = \qty[j(j + 1) - n(n + 1)]u_j\label{legendre-polynominal-u-expantion-recurr}
      \end{align}
      なる漸化式が成立する.
      \refe{legendre-polynominal-u-expantion-recurr}において$j = n$を代入すると,$u_{n + 2} = 0$となる.
      また,$j = n + 1$を代入すると\refe{legendre-polynominal-u-expantion-np1}より$u_{n + 1} = 0$である.
      よって,
      \begin{align}
        0 = u_{n + 1} = u_{n + 2} = u_{n + 3} = u_{n + 4} = \cdots 
      \end{align}
      となる.また,\refe{legendre-polynominal-u-expantion-recurr}に$j = n - 2$を代入すると,
      \begin{align}
        u_{n - 2} = -\frac{n(n - 1)}{2(2n - 1)}u_n\label{legendre-polynominal-u-expantion-recurr-2}
      \end{align}
      となる.よって,\refe{legendre-polynominal-u-expantion-n}と\refe{legendre-polynominal-u-expantion-recurr-2}を用いて
      \refe{legendre-polynominal-u-expantion}を表すと,
      \begin{align}
        P_n(x) &= \frac{(2n)!}{2^n(n!)^2}\qty[x^n - \frac{n(n - 1)}{2(2n - 1)}x^{n - 2} + \frac{n(n - 1)(n - 2)(n - 3)}{2\cdot 4\cdot (2n - 1)(2n - 3)}x^{n - 4} + \cdots] \\ 
        &= \sum_{s = 0}^{\lfloor n/2 \rfloor}(-1)^s \frac{(2n - 2s)!}{2^ns!(n - s)!(n - 2s)!}x^{n - 2s}
      \end{align}
      となる.
      \begin{myprop}{Rodriguesの公式}{}
        Legendre多項式$P_n(x)$は,
        \begin{align}
          P_n(x) &= \frac{1}{2^nn!}\dv[n]{x}\qty(x^2 - 1)^n\label{legendre-pn-easy}
        \end{align}
        とも書ける.
        \tcblower
        \begin{proof}
          \begin{align}
            \dv[n]{x}\qty(x^2 - 1)^n &= \sum_{s = 0}^{n}\dv[n]{x}(-1)^s\mqty(n \\ k)x^{2n - 2k} \\ 
            &= \sum_{s = 0}^{\lfloor n/2 \rfloor}(-1)^s\frac{n!}{s!(n - s)!}\frac{(2n - 2s)!}{(n - 2k)!}
          \end{align}
          なる関係を用いると,WIP
        \end{proof}
      \end{myprop}
    \section{Legendre陪多項式}
      \begin{mydef}{Legendre陪多項式}{}
        Legendre陪多項式$P_n^m(x)$は$m \leq n$として,
        \begin{align}
          P_n^m(x) &\coloneqq \qty(1 - x^2)^{m/2}\dv[m]{x}P_n(x)
        \end{align}
        と定義される.
      \end{mydef}
      \begin{myrem}{}{}
        \refe{legendre-pn-easy}を用いれば,
        \begin{align}
          P_n^m(x) &= \qty(1 - x^2)^{m/2}\dv[m]{x}P_n(x) \\ 
          &= \qty(1 - x^2)^{m/2}\dv[m]{x}\frac{1}{2^nn!}\dv[n]{x}\qty(x^2 - 1)^n \\ 
          &= \frac{1}{2^nn!}\qty(1 - x^2)^{m/2}\dv[n + m]{x}\qty(x^2 - 1)^n \label{assos-legendre-polynominal}
      \end{align}
      と書ける.
      \end{myrem}
      \begin{myprop}{Legendreの陪微分方程式}{}
        $P_n^m$はLegendreの陪微分方程式,
        \begin{align}
          \mathcal{L}_mP_n^m(x) + n(n + 1)P_n^m(x) = 0\label{assos-legendre-diff-eq}
        \end{align}
        を満たす.
        \tcblower
        WIP
      \end{myprop}
      Legendre多項式の直交性とノルムを調べる.
      Legendre陪多項式の直交性は$\mathcal{L}_m$がHermite演算子であり,その固有函数である$P_n^m(x)$が直交することより従う.
      自分自身との内積,つまり,$\inner{P_n^m(x)}{P_n^m(x)}$の値を計算する.
      \refe{assos-legendre-polynominal}を用いて,\refe{inner-prod-legendre}で示した内積の定義に従って計算する.
      $n + m$回部分積分を行うと,
      \begin{align}
        \inner{P_{n}^m(x)}{P_{n}^m(x)} &= \int_{-1}^{1}\frac{1}{2^{2n}\qty(n!)^2}\qty(1 - x^2)^m \qty{\dv[n + m]{x}\qty(x^2 - 1)^m}\qty{\dv[n + m]{x}\qty(x^2 - 1)^m} \dd{x} \\ 
        &= \frac{1}{2^{2n}\qty(n!)^2}\int_{-1}^{1}\qty(1 - x^2)^m \qty{\dv[n + m]{x}\qty(x^2 - 1)^m}\dv{x}\qty{\dv[n + m - 1]{x}\qty(x^2 - 1)^m} \dd{x} \\ 
        &= \frac{1}{2^{2n}\qty(n!)^2}\qty[\qty(1 - x^2)^m\qty{\dv[n + m]{x}\qty(x^2 - 1)^m}\qty{\dv[n + m - 1]{x}\qty(x^2 - 1)^m}]_{-1}^{1} \\ 
        &\ \ -\frac{1}{2^{2n}\qty(n!)^2} \int_{-1}^{1}\qty{\dv{x}\qty(1 - x^2)^m\qty{\dv[n + m]{x}(x^2 - 1)^n}}\qty{\dv[n + m + 1]{x}\qty(x^2 - 1)^m}\dd{x} \\ 
        &= -\frac{1}{2^{2n}\qty(n!)^2}\int_{-1}^{1}\qty{\dv{x}\qty(1 - x^2)^m\qty{\dv[n + m]{x}(x^2 - 1)^n}}\qty{\dv[n + m + 1]{x}\qty(x^2 - 1)^m} \dd{x} \\
        &= \cdots \\ 
        &= \frac{\qty(-1)^{n + m}}{2^{2n}\qty(n!)^2}\int_{-1}^{1}\qty(x^2 - 1)^n\qty{\dv[n + m]{x}\qty(1 - x^2)^m\qty{\dv[n + m]{x}\qty(x^2 - 1)^n}} \dd{x} \\ 
        &= \frac{\qty(-1)^{n + m}}{2^{2n}\qty(n!)^2}\int_{-1}^{1}\qty(x^2 - 1)^n\qty[\sum_{k = 0}^{n + m}\mqty(n + m \\ k)\qty{\dv[n + m - k]{x}\qty(1 - x^2)^m}\qty{\dv[n + m + k]{x}\qty(x^2 - 1)^n}] \dd{x}\label{use-leibniz-formula}
      \end{align}
      となる.最終行でLeibnizの公式を用いた.
      \refe{use-leibniz-formula}の和の中の$n + m - k$階微分と$n + m - k$階微分を考える.
      $\qty(1 - x^2)^m$と$(x^2 - 1)^n$の最高次数は,それぞれ$2m$と$2n$であるから,$2m \geq n + m - k$かつ$2n \geq n + m + k$なる$k$でのみ和の中は0でなくなる.
      つまり,$n - m\leq k$かつ$n - m \geq k$なる$k$は$k = n - m$のみである.
      よって,\refe{use-leibniz-formula}は,
      \begin{align}
        \inner{P_{n}^m(x)}{P_{n}^m(x)} &= \frac{\qty(-1)^{n + m}}{2^{2n}\qty(n!)^2}\int_{-1}^{1}\qty(x^2 - 1)^n\qty[\sum_{k = 0}^{n + m}\mqty(n + m \\ k)\qty{\dv[n + m - k]{x}\qty(1 - x^2)^m}\qty{\dv[n + m + k]{x}\qty(x^2 - 1)^n}] \dd{x} \\ 
        &= \frac{\qty(-1)^{n + m}}{2^{2n}\qty(n!)^2}\int_{-1}^{1}\qty(x^2 - 1)^n\qty[\mqty(n + m \\ n - m)\qty{\dv[2m]{x}\qty(1 - x^2)^m}\qty{\dv[2n]{x}\qty(x^2 - 1)^n}] \dd{x} \\ 
        &= \frac{\qty(-1)^{n + m}}{2^{2n}\qty(n!)^2}(-1)^m\qty(2m)!\qty(2n)!\frac{(n + m)!}{(n - m)!(2m)!}\int_{-1}^{1}\qty(x^2 - 1)^n\dd{x} 
      \end{align}
      積分は,$x = \cos\theta$と置換すると,
      \begin{align}
        \int_{-1}^{1}\qty(x^2 - 1)^n\dd{x} = (-1)n\int_{-1}^{1}\sin^{2n + 1}\theta\dd{\theta}
      \end{align}
      となる.$I_{2n + 1}$を,
      \begin{align}
        I_{2n + 1} \coloneqq \int_{0}^{\pi}\sin^{2n + 1}\theta\dd{\theta}
      \end{align}
      と定義すると,
      \begin{align}
        I_{2n + 1} &= \int_{0}^{\pi} \sin^{2n}\theta\dv{\theta}(-\cos\theta)\dd{\theta} \\ 
        &= \qty[\sin^{2n}\theta\cdot(-\cos\theta)]_{0}^{\pi} + 2n\int_{0}^{\pi}\sin^{2n - 1}\theta \cos^2\theta\dd{\theta} \\ 
        &= 2nI_{2n - 1} - 2nI_{2n + 1}
      \end{align}
      となるので,
      \begin{align}
        I_{2n + 1} &= \frac{2n}{2n + 1}I_{2n - 1} \\ 
        &= \frac{2n}{2n + 1}\frac{2n - 2}{2n - 1}I_{2n - 3} \\ 
        &= \cdots \\ 
        &= \frac{(2n)!!}{(2n + 1)!!}I_1 \\ 
        &= 2\frac{(2n)!!}{(2n + 1)!!} \\ 
        &= 2\cdot 2^nn! \cdot \frac{2^nn!}{(2n + 1)!}
      \end{align}
      となる.2重階乗について,
      \begin{align}
        (2n)!! &= 2^nn! \\ 
        (2n + 1) !! &= \frac{(2n + 1)!}{(2n)!!} = \frac{(2n + 1)!}{2^nn!}
      \end{align}
      なる関係が成り立つことを用いると,
      \begin{align}
        \inner{P_{n}^m(x)}{P_{n}^m(x)} &= \frac{\qty(-1)^{n + m}}{2^{2n}\qty(n!)^2}(-1)^m\qty(2m)!\qty(2n)!\frac{(n + m)!}{(n - m)!(2m)!}(-1)^n2\frac{(2n)!!}{(2n + 1)!!} \\
        &=  \frac{\qty(-1)^{n + m}}{2^{2n}\qty(n!)^2}(-1)^m\qty(2m)!\qty(2n)!\frac{(n + m)!}{(n - m)!(2m)!}(-1)^n2\cdot 2^nn! \cdot \frac{2^nn!}{(2n + 1)!} \\
        &= \frac{2}{2n + 1}\frac{(n + m)!}{(n - m)!}
      \end{align}
      となる.
      Legendre多項式の直交性とまとめて書くと,
      \begin{align}
        \inner{P_{n'}^m(x)}{P_{n}^m(x)} = \delta_{n}^{n'}\frac{2}{2n + 1}\frac{(n + m)!}{(n - m)!}\label{pnm-norm}
      \end{align}
      となる.
    \section{Bessel函数}
      \begin{mydef}{$\mathcal{L}_{\nu}$の定義}{}
        函数空間は,$\qty[0, a]$で$C^1$級の複素函数全体であるとする.
        $\mathcal{L}_{\nu}$を,
        \begin{align}
          \mathcal{L}_{\nu} &\coloneqq \frac{1}{z}\qty[\dv{z}\qty{z\dv{z}} - \frac{\nu^2}{z}]
        \end{align}
        と定義する.
      \end{mydef}
      \begin{myrem}{}{}
        $\mathcal{L}_{\nu}$は\refe{strum-liouville-operator-def}において,
        また,
        \begin{align}
          a &\to 0 \\ 
          b &\to a \\
          \rho(z) &= x \\ 
          p(z) &= x \\ 
          q(z) &= -\frac{\nu^2}{z},\ \nu \geq 0
        \end{align}
        としたものである.
        すなわち,
        \begin{align}
          \inner{f}{g} &= \int_{0}^{a}f^*(z)g(z)x\dd{z}
        \end{align}
        である.
      \end{myrem}
      さて,
      \begin{align}
        \mathcal{L}_{\nu}J_{\nu}(z) &= - J_{\nu}(z)  \\ 
        \Leftrightarrow \qty(\dv[2]{z} + \frac{1}{z}\dv{z} - \frac{\nu^2}{z^2})J_{\nu}(z) &= -J_{\nu}(z) \label{bessel-differential-eq-sub}\\ 
        \Leftrightarrow \dv[2]{z}J_{\nu}(z) + \frac{1}{z}\dv{z}J_{\nu}(z) - \frac{\nu^2}{z^2}J_{\nu}(z) &= -J_{\nu}(z) \label{bessel-differential-eq}
      \end{align}
      なる$J_{\nu}(z)$を考える.1階微分の項と微分をしない項は$z = 0$で発散するが,それぞれ$z$と$z^2$をかければ発散しないので確定特異点である.
      このとき,$z = 0$の周りで$J_{\nu}(z)$を指数$\alpha$のFrobenius展開をすると,
      \begin{align}
        J_{\nu}(z) = z^{\alpha}\sum_{n}u_nz^n\label{bessel-func-frobenius}
      \end{align}
      となる.
      ただし,計算の簡単のために$n < 0$なる任意の$n$に対して$u_n = 0$と定めた.
      また,$u_0 \neq 0$とする.
      \refe{bessel-func-frobenius}を\refe{bessel-differential-eq}に代入することを考える.
      $J_{\nu}(z)$の1階微分と2階微分が,
      \begin{align}
        \dv{z}J_{\nu}(z) &= \sum_{n}(n + \alpha)u_nz^{n + \alpha - 1} \\ 
        \dv[2]{z}J_{\nu}(z) &= \sum_{n}(n + \alpha)(n + \alpha - 1)u_nz^{n + \alpha - 2} 
      \end{align}
      と書けることを用いると,
      \begin{align}
        \sum_{n}\qty[(n + \alpha)(n + \alpha - 1) + (n + \alpha) - \nu^2]u_nz^{n + \alpha - 2} &= -\sum_{n}u_nz^{n + \alpha} \\ 
        \Leftrightarrow \sum_{n}\qty[(n + \alpha)(n + \alpha - 1) + (n + \alpha) - \nu^2]u_nz^{n + \alpha - 2} &= -\sum_{n}u_{n - 2}z^{n + \alpha -2} 
      \end{align}
      となる.係数を比較すると,
      \begin{align}
        \qty[(n + \alpha)^2 - \nu^2]u_n = -u_{n - 2}\label{bessel-func-recurr}
      \end{align}
      となる.
      \refe{bessel-func-recurr}に$n = 1$を代入すると$u_{-1} = 0$より,
      \begin{align}
        0 = u_1 = u_3 = \cdots
      \end{align}
      を得る.
      また,$n = 0$を代入すると$u_0 \neq 0$より,
      \begin{align}
        (0 + \alpha)^2 - \nu^2 &= 0 \\ 
        \Rightarrow \alpha = \pm \nu
      \end{align}
      となる.
      $\alpha = \nu$を採用して\refe{bessel-func-recurr}を用いると,
      \begin{align}
        u_{n + 2} = -\frac{1}{(n + \nu + 2)^2 - \nu^2}u_n 
      \end{align}
      を得る.
      $\nu \in \mathbb{Z}$のときは,
      \begin{align}
        J_{\nu}(z) &= u_0z^{\nu}\qty(1 - \frac{1}{2(2\nu + 2)}z^2 + \frac{1}{2\cdot 4(2\nu + 2)(2\nu + 4)}z^4 + \cdots + (-1)^n\frac{(2\nu)!!}{(2n)!!(2\nu + 2n)!!}z^{2n} + \cdots) \\ 
        &= u_0z^{\nu}\sum_{n = 0}(-1)^n\frac{2^{\nu}\nu!}{2^nn!(\nu + n)!2^{\nu + n}} z^{2n} \\ 
        &= u_0\nu!2^{\nu}\sum_{n = 0}\frac{1}{n!(\nu + n)!}\qty(\frac{z}{2})^{\nu + 2n}
      \end{align}
      となる.$\nu \notin \mathbb{Z}$のときも,表せるようにガンマ函数を用いて一般化する.
      \refe{bessel-differential-eq}の形より,明らかに定数倍が許容されるから,
      \begin{align}
        u_0 = \frac{1}{2^{\nu}\Gamma(\nu + 1)}
      \end{align}
      となるように$u_0$を定めておくと,
      \begin{align}
        J_{\nu}(z) &= \sum_{n}\frac{(-1)^n}{n!\Gamma(\nu + k + 1)}\qty(\frac{z}{2})^{\nu + 2n}
      \end{align}
      を得る.$J_{\nu}(z)$をBessel函数という.
      また,$\alpha = -\nu$を採用したときは,
      \begin{align}
        J_{-\nu}(z) &= \sum_{n}\frac{(-1)^n}{n!\Gamma(-\nu + k + 1)}\qty(\frac{z}{2})^{-\nu + 2n}
      \end{align}
      となり,これは$\nu \notin \mathbb{Z}$のときに$J_{\nu}(z)$と独立な解となることが示せる.
      $J_{-\nu}(z)$をNeumann函数という.
      \par
      得られたBessel函数とNeumann函数を用いて,
      \begin{align}
        j_n(z) &\coloneqq \sqrt{\frac{\pi}{2z}}J_{n + \frac{1}{2}}(z) \\ 
        y_n(z) &\coloneqq (-1)^{n + 1}\sqrt{\frac{\pi}{2z}}J_{-n-\frac{1}{2}}(z) \\ 
      \end{align}
      を定義する.$j_n(z)$と$y_n(z)$はそれぞれ,球Bessel函数,球Neumann函数という.
      さらに,球Bessel函数と球Neumann函数を用いて,
      \begin{align}
        h_n^{(1)}(z) &\coloneqq j_n(z) + \i y_n(z) \\ 
        h_n^{(2)}(z) &\coloneqq j_n(z) - \i y_n(z)
      \end{align}
      を定義する.$h_n^{(1)}$と$h_n^{(2)}$はそれぞれ,第1種Hankel函数,第2種Hankel函数という.
  \chapter{物理現象}
    \section{Schr\"odinger方程式}
      1次元調和振動子モデルを考えよう.
      Schr\"odinger方程式は,
      \begin{align}
        \qty(-\frac{\hbar}{2m}\dv[2]{x} + \frac{1}{2}m\omega^2x^2)\psi(x) &= E\psi(x)\label{schrodinger-eq-wave-func}
      \end{align}
      である.
      いささか唐突だが,波動函数が,
      \begin{align}
        \psi(x) &= f(s)\exp\qty(-\frac{s^2}{2}) \label{psi-fs}\\ 
        s &\coloneqq \sqrt{\frac{m\omega}{\hbar}}x\label{s-def}
      \end{align}
      と書けたとする.$f(s)$がHermite多項式となることを示す.
      \par
      \refe{schrodinger-eq-wave-func}の両辺を$-\frac{\hbar^2}{2m}$で割って,$x$から$s$に変数変換\footnote{
        この変数変換は$x$の無次元化ともとらえられる.
        実際に\refe{s-def}の右辺の次元を調べると,
        \begin{align*}
          \sqrt{\frac{\r{kg\cdot s^{-1}}}{\r{kg\cdot m^2\cdot s^{-2}\cdot s}}}\r{m} = 1
        \end{align*}
        である.
      }すると,
      \begin{align}
        \qty(-\frac{\hbar}{2m}\dv[2]{x} + \frac{1}{2}m\omega^2x^2)\psi(x) &= E\psi(x) \\ 
        \Leftrightarrow \qty[\dv[2]{x} - \frac{m^2\omega^2}{\hbar^2}x^2 + \frac{2mE}{\hbar}]\psi(x) &= 0 \\ 
        \Leftrightarrow \qty[\qty(\dv{s}{x})^2\dv[2]{s} - \frac{\hbar}{m\omega}s^2 + \frac{2mE}{\hbar^2}]f(s)\exp\qty(-\frac{s^2}{2}) &= 0 \\ 
        \Leftrightarrow \qty[\frac{m\omega}{\hbar}\dv[2]{s} - \frac{\hbar}{m\omega}s^2 + \frac{2mE}{\hbar^2}]f(s)\exp\qty(-\frac{s^2}{2}) &= 0 \\ 
        \Leftrightarrow \qty(\dv[2]{s} - s^2 + \frac{2E}{\hbar\omega})f(s)\exp\qty(-\frac{s^2}{2}) &= 0 \label{before-hermite-polynominal}
      \end{align}
      と書ける.第1項について,$\dv[2]{s}f(s)\exp\qty(-\frac{s^2}{2})$を計算しよう.
      Leibniz則より,
      \begin{align}
        \dv[2]{s}f(s)\exp\qty(-\frac{s^2}{2}) &= \dv[2]{f}{s}\exp\qty(-\frac{s^2}{2}) + 2\dv{f}{s}\dv{s}\qty(\exp\qty(-\frac{s^2}{2})) + f(s)\dv[2]{s}\qty(\exp\qty(-\frac{s^2}{2})) \\ 
        &= \dv[2]{f}{s}\exp\qty(-\frac{s^2}{2}) - 2s\dv{f}{s}\exp\qty(-\frac{s^2}{2}) + f(s)\qty(s^2 - 1)e\exp\qty(-\frac{s^2}{2}) \\ 
        &= \qty(\dv[2]{s} - 2s\dv{s} + \qty(s^2 - 1))f(s)\exp\qty(-\frac{s^2}{2})
      \end{align}
      と計算できるから,\refe{before-hermite-polynominal}は,
      \begin{align}
        \qty(\dv[2]{s} - s^2 + \frac{2E}{\hbar\omega})f(s)\exp\qty(-\frac{s^2}{2}) &= 0 \\ 
        \Leftrightarrow \qty(\dv[2]{s} - 2s\dv{s} + \qty(s^2 - 1) - s^2 + \frac{2E}{\hbar\omega})f(s)\exp\qty(-\frac{s^2}{2}) &= 0 \\ 
        \Rightarrow \qty(\dv[2]{s} - 2s\dv{s} + \frac{2E}{\hbar\omega} - 1)f(s) &= 0
      \end{align}
      となる.
      Hermite多項式の形は,
      \begin{align}
        \qty(\dv[2]{s} - 2s\dv{s} + 2n)H_n(s) = 0
      \end{align}
      であったから,
      \begin{align}
        \frac{2E}{\hbar\omega} - 1 &= 2n \\ 
        f(s) &\to H_n(s)
      \end{align}
      とすれば良いことがわかる.
      $n$は非負整数で,$n = 0$では零点振動に対応する.
      規格化定数を$A$とすれば,波動函数は,
      \begin{align}
        \psi_n(x) = AH_n(s)\exp\qty(-\frac{s^2}{2})
      \end{align}
      と書ける.
      規格化定数は,
      \begin{align}
        1 &= \int\dd{x}\abs{\psi(x)}^2 \\ 
        &= \abs{A}^2\int\dd{x}H_n(s)H_n(s)\e^{-s^2} \\ 
        &= \abs{A}^2\sqrt{\frac{\hbar}{m\omega}}\int\dd{s}H_n(s)H_n(s)\e^{-s^2} \\ 
        &= \abs{A}^2\sqrt{\frac{\hbar}{m\omega}}\sqrt{\pi}2^nn! 
      \end{align}
      より,
      \begin{align}
        A = \sqrt{\frac{1}{2^nn!}\sqrt{\frac{m\omega}{\pi\hbar}}}
      \end{align}
      となる.波動函数は,
      \begin{align}
        \psi(x) = \sqrt{\frac{1}{2^nn!}\sqrt{\frac{m\omega}{\pi\hbar}}}H_n\qty(\sqrt{\frac{m\omega}{\hbar}}x)\exp\qty(-\frac{m\omega}{2\hbar}x^2)
      \end{align}
      となる.
    \section{Helmholtz方程式}
      Helmholtz方程式,
      \begin{align}
        \qty(\laplacian + \kappa^2)u(r, \theta, \phi) = 0\label{helmholtz-eq-wave}
      \end{align}
      を考える.
      Helmholtzが登場する物理系を少し検討しよう.
      % 散乱問題の話したいけど,それより前に角運動量演算子の固有函数って言わないと...
      $u(\bm{r})$を直交基底である$\e^{\i m\phi}$で展開すると,
      \begin{align}
        A_m(r, \theta) &\coloneqq \int_{0}^{2\pi}u(r, \theta, \phi)\e^{-\i m\phi}\dd{\phi} \\ 
        u(r, \theta, \phi) &= \sum_{m}A_m(r, \theta)\e^{\i m\phi}\label{u-expand-with-a} \\ 
      \end{align}
      となる.
      また,極座標ラプラシアンは,
      \begin{align}
        \laplacian = \frac{1}{r^2}\pdv{r}\qty(r^2\pdv{r}) + \frac{1}{r^2\sin\theta}\pdv{\theta}\qty(\sin\theta\pdv{\theta}) + \frac{1}{r^2\sin^2\theta}\pdv[2]{\phi} \label{polar-laplacian}
      \end{align}
      である.
      \refe{helmholtz-eq-wave}に\refe{polar-laplacian}を用いて,\refe{u-expand-with-a}を代入すると,
      \begin{align}
        &\frac{1}{r^2}\pdv{r}\qty(r^2\pdv{r}\sum_{m}A_m(r, \theta)\e^{\i m\phi}) + \frac{1}{r^2\sin\theta}\pdv{\theta}\qty(\sin\theta\pdv{\theta}\sum_{m}A_m(r, \theta)\e^{\i m\phi}) + \frac{1}{r^2\sin^2\theta}\pdv[2]{\phi}\sum_{m}A_m(r, \theta)\e^{\i m\phi} \notag \\ 
        &+ \kappa^2\sum_{m}A_m(r, \theta)\e^{\i m\phi} = 0 \\ 
        \Leftrightarrow &\sum_{m}\e^{\i m\phi}\qty[
          \frac{1}{r^2}\pdv{r}\qty(r^2\pdv{r}A_m(r, \theta)) + 
          \frac{1}{r^2\sin\theta}\pdv{\theta}\qty(\sin\theta\pdv{\theta}A_m(r, \theta)) 
          - \frac{m^2}{r^2\sin^2\theta}A_m(r, \theta) + \kappa^2A_m(r, \theta)
        ] = 0 \label{am-before-int}
      \end{align}
      となる.
      \refe{am-before-int}に左から$\int_{0}^{2\pi}\dd{\phi}\e^{-\i m\phi}$をかける,すなわち,$\e^{\i m\phi}$に射影すると,
      \begin{align}
        \frac{1}{r^2}\pdv{r}\qty(r^2\pdv{r}A_m(r, \theta)) + \frac{1}{r^2\sin\theta}\pdv{\theta}\qty(\sin\theta\pdv{\theta}A_m(r, \theta)) - \frac{m^2}{r^2\sin^2\theta}A_m(r, \theta) + \kappa^2A_m(r, \theta) = 0 \label{a-condition}
      \end{align}
      を得る.
      Legendre陪多項式を定義するときに用いた$\mathcal{L}_m$において,$x = \cos\theta$とすると,
      \begin{align}
        \mathcal{L}_m &= \dv{x}\qty{(1 - \cos^2\theta)\dv{x}} - \frac{m^2}{1 - \cos^2\theta} \\ 
        &= \dv{x}\qty{\sin^2\theta\dv{x}} - \frac{m^2}{\sin^2\theta} \\ 
        &= \dv{\theta}{x}\dv{x}\qty{\sin^2\theta\dv{\theta}{x}\dv{x}} - \frac{m^2}{\sin^2\theta} \\ 
        &= -\frac{1}{\sin\theta}\dv{x}\qty{\sin^2\theta\qty(-\frac{1}{\sin\theta})} - \frac{m^2}{\sin^2\theta} \\ 
        &= \frac{1}{\sin\theta}\dv{x}\qty{\sin\theta\dv{\theta}} - \frac{m^2}{\sin^2\theta}
      \end{align}
      であるから,\refe{a-condition}は,
      \begin{align}
        \frac{1}{r^2}\pdv{r}\qty(r^2\pdv{r}A_m(r, \theta)) + \frac{1}{r^2}\mathcal{L}_{\abs{m}}A_m(r, \theta) + \kappa^2 A_m(r, \theta) = 0 \label{a-condition-use-lm}
      \end{align}
      となる.
      \par
      続いて,$A_m(r, \theta)$を直交基底であるLegendre陪多項式を用いて,
      \begin{align}
        A_m(r, \theta) = \sum_{n = \abs{m}}^{\infty}B_{nm}(r)P_{n}^{\abs{m}}(\cos\theta) \label{am-expand-with-pnm}
      \end{align}
      と展開する.
      \refe{a-condition-use-lm}の両辺に$P_{n}^{\abs{m}}(\cos\theta)$との内積をとる.
      \refe{pnm-norm}より,
      \begin{align}
        \inner{P_n^{m}(\cos\theta)}{P_n^{m}(\cos\theta)} = \delta_{n}^{n'}\frac{2}{2n + 1}\frac{(n + m)!}{(n - m)!}
      \end{align}
      であることと,\refe{assos-legendre-diff-eq}より,
      \begin{align}
        \mathcal{L}_mP_n^m(x) + n(n + 1)P_n^m(x) = 0\label{angular-momentum-prep}
      \end{align}
      であることを用いれば,
      \begin{align}
        & \inner{P_n^{m}(\cos\theta)}{\frac{1}{r^2}\pdv{r}\qty(r^2\pdv{r}A_m(r, \theta)) + \frac{1}{r^2}\mathcal{L}_{\abs{m}}A_m(r, \theta) + \kappa^2 A_m(r, \theta)} = 0 \\ 
        \Leftrightarrow & \frac{1}{r^2}\inner{P_n^{m}(\cos\theta)}{\pdv{r}\qty(r^2\pdv{r}A_m(r, \theta))} + \frac{1}{r^2}\inner{P_n^{m}(\cos\theta)}{\mathcal{L}_{\abs{m}}A_m(r, \theta)} + \inner{P_n^{m}(\cos\theta)}{\kappa^2 A_m(r, \theta)} = 0 \\ 
        \Leftrightarrow & \frac{1}{r^2}\qty{\dv{r}r^2\qty{\dv{r}B_{nm}(r)}}\inner{P_n^{m}(\cos\theta)}{P_n^{m}(\cos\theta)} + \frac{1}{r^2}B_{nm}(r)\inner{P_n^{m}(\cos\theta)}{\mathcal{L}_{\abs{m}}P_n^{m}(\cos\theta)} \notag \\ 
        &+ \kappa^2B_{nm}(r)\inner{P_n^{m}(\cos\theta)}{P_n^{m}(\cos\theta)} = 0 \\ 
        \Leftrightarrow&  \qty(\frac{1}{r^2}\qty{\dv{r}r^2\qty{\dv{r}B_{nm}(r)}} - \frac{1}{r^2}n(n + 1)B_{nm}(r) + \kappa^2 B_{nm}(r))\inner{P_n^{m}(\cos\theta)}{P_n^{m}(\cos\theta)} = 0 \\ 
        \Rightarrow& \frac{1}{r^2}\qty{\dv{r}r^2\qty{\dv{r}B_{nm}(r)}} - \frac{1}{r^2}n(n + 1)B_{nm}(r) + \kappa^2 B_{nm}(r) = 0\label{bnm-condition}
      \end{align}
      を得る.
      \par
      \refe{bnm-condition}において,
      \begin{align}
        \rho \coloneqq \kappa r\label{rho-def}
      \end{align}
      と変数変換して,
      \begin{align}
        C_{nm}(\rho) \coloneq \sqrt{\rho}B_{nm}(r)\label{cnm-def}
      \end{align}
      と定義して代入すれば,
      \begin{align}
        \qty{\dv[2]{\rho} + \frac{1}{\rho}\dv{\rho} + 1 - \frac{1}{\rho^2}\qty(n + \frac{1}{2})^2}C_{nm}(\rho) = 0\label{bessel-form-rho}
      \end{align}\
      を得る.
      \refe{bessel-form-rho}と\refe{bessel-differential-eq-sub}である,
      \begin{align}
        \mathcal{L}_{\nu}J_{\nu}(z) &= - J_{\nu}(z)  \\ 
        \Leftrightarrow \qty(\dv[2]{z} + \frac{1}{z}\dv{z} + 1 - \frac{\nu^2}{z^2}) &= 0 \label{bessel-differential-eq-sub-2}
      \end{align}
      を比べると,\refe{bessel-differential-eq-sub-2}において,
      \begin{align}
        z &\to \rho \\ 
        \nu &= n + \frac{1}{2}
      \end{align}
      としたものであると分かる.
      つまり,$C_{nm}(\rho)$はBessel函数$J_{n + \frac{1}{2}}(\rho)$とNeumann函数$J_{-n - \frac{1}{2}}(\rho)$の線型結合で書かれるので,
      係数をそれぞれ$a''_{nm}$,$b''_{nm}$として,
      \begin{align}
        C_{nm}(\rho) = a''_{nm}J_{n + \frac{1}{2}}(\rho) + b''_{nm}J_{-n - \frac{1}{2}}(\rho)\label{expand-cnm-ans-tmp}
      \end{align}
      と書く.
      今後の変形のために,球Bessel函数$j_n(\rho)$と球Neumann函数$y_n(\rho)$を用いて\refe{expand-cnm-ans-tmp}を書き直せば,
      \begin{align}
        C_{nm}(\rho) &= a''_{nm}J_{n + \frac{1}{2}}(\rho) + b''_{nm}J_{-n - \frac{1}{2}}(\rho) \\ 
        &= \sqrt{\frac{2\rho}{\pi}}a''_{nm}j_n(\rho) + \sqrt{\frac{2\rho}{\pi}}b''_{mn}y_n(\rho) \\ 
        &= \sqrt{\rho}\qty[a'_{nm}j_n(\rho) + b'_{mn}y_n(\rho)] \label{expand-cnm-ans}
      \end{align}
      である.ただし,
      \begin{align}
        a'_{nm} &\coloneq a''_{nm}\sqrt{\frac{2}{\pi}} \\ 
        b'_{nm} &\coloneq b''_{nm}\sqrt{\frac{2}{\pi}}
      \end{align}
      と定めた.
      \par
      さて,今Helmholtz方程式の解の基底展開を行っているのであった.
      今までに定義した展開を全てまとめると,
      \begin{align}
        \text{\refe{u-expand-with-a}より}&\ u(r, \theta, \phi) = \sum_{m = -\infty}^{\infty} A_m(r, \theta) \e^{\i m\phi} \\ 
        \text{\refe{am-expand-with-pnm}より}&\ A_m(r, \theta) = \sum_{n = \abs{m}}^{\infty}B_{nm}(r)P_{n}^{\abs{m}}(\cos\theta) \\ 
        \text{\refe{rho-def}より}&\ \rho = \kappa r \\ 
        \text{\refe{cnm-def}より}&\ B_{nm}(r) = \frac{1}{\sqrt{\rho}}C_{nm}(\rho) \\ 
        \text{\refe{expand-cnm-ans}より}&\ C_{nm}(\rho) = \sqrt{\rho}\qty[a'_{nm}j_n(\rho) + b'_{mn}y_n(\rho)]
      \end{align}
      となる.下から上に全て代入して整理すると,
      \begin{align}
        u(r, \theta, \phi) = \sum_{m = -\infty}^{\infty}\sum_{n = \abs{m}}^{\infty}\qty[a'_{nm}j_n(\kappa r) + b'_{nm}y_n(\kappa r)] P_{n}^{\abs{m}}(\cos\theta)\e^{\i m\phi} \label{u-expand-form1}
      \end{align}
      となる.
      今後の計算を行うとき,
      \begin{align}
        \inner{\e^{\i m'\phi}}{\e^{\i m\phi}} &= 2\pi\delta_m^{m'} \\ 
        \inner{P_{n'}^m(x)}{P_n^m(x)} &= \delta_{n}^{n'}\frac{2}{2n + 1}\frac{(n + m)!}{(n - m)!}
      \end{align}
      のように規格化されていないのはいささか都合が悪いので,球面分布函数$Y_{nm}(\theta, \phi)$を,
      \begin{align}
        Y_{nm}(\theta, \phi) &\coloneqq (-1)^m\qty(\frac{1}{2\pi})^{1/2}\qty(\frac{2n + 1}{2}\frac{(n - m)!}{(n + m)!})^{1/2}P_n^m(\cos\theta)\e^{\i m\phi} \\ 
        &= (-1)^m\sqrt{\frac{2n + 1}{4\pi}\frac{(n - m)!}{(n + m)!}}P_n^m(\cos\theta)\e^{\i m\phi}
      \end{align}
      として定義しておく.
      球面分布函数を用いて\refe{u-expand-form1}を書き換えれば,定数倍は$a'_{nm}$や$b'_{nm}$で吸収することにして,
      \begin{align}
        u(r, \theta, \phi) = \sum_{m = -\infty}^{\infty}\sum_{n = \abs{m}}^{\infty}\qty[a_{nm}j_n(\kappa r) + b_{nm}y_n(\kappa r)]Y_{nm}(\theta, \phi)\label{u-expand-form2}
      \end{align}
      となる.
      当然,
      \begin{align}
        a_{nm} &\coloneqq \qty[(-1)^m\sqrt{\frac{2n + 1}{4\pi}\frac{(n - m)!}{(n + m)!}}]^{-1}a'_{nm} \\ 
        b_{nm} &\coloneqq \qty[(-1)^m\sqrt{\frac{2n + 1}{4\pi}\frac{(n - m)!}{(n + m)!}}]^{-1}b'_{nm}
      \end{align}
      である.
    \end{document}